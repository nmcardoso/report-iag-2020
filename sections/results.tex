\section*{Resultados Parciais}
O modelo 2 (figura \ref{fig:pretrained_model}) usa uma rede pré-treinada VGG16 com os pesos da ImageNet. Por ela ter muitas camadas de convolução e ser treinada com uma quantidade muito grande de amostras, é considerável usá-la, pois ela tem muitos filtros para identificar formas, cores e texturas em imagens.

\pagebreak
\begin{figure}[h!]
  \centering
  \includegraphics[width=.78\textwidth]{figures/conv_train.pdf}
  \caption{Gráfico da acurácia e perda em cada época de treinamento do modelo 1}
  \label{fig:conv_train}
\end{figure}

\begin{figure}[h!]
  \centering
  \includegraphics[width=.78\textwidth]{figures/pretrained_train.pdf}
  \caption{Gráfico da acurácia e perda em cada época de treinamento do modelo 2}
  \label{fig:pretrained_train}
\end{figure}