\section{Resultados}

A tabela \ref{tab:eval} mostra a acurácia do modelo classificando imagens do conjunto de teste.

\begin{table}[h!]
  \centering

  \caption{Acurácia no conjunto de teste para os modelos.}
  \label{tab:eval}

  \begin{tabular}{cc}
    \toprule
    Modelo & Acurácia no conjunto de teste \\
    \midrule
    1      & 85,32\%                       \\
    2      & 86,81\%                       \\
    3      & 83,88\%                       \\
    \bottomrule
  \end{tabular}
\end{table}


Uma das formas de inspecionar o aprendizado de uma rede neural artificial é pela análise dos gráficos \texttt{Acurácia x Época} e \texttt{Perda x Época}, como das figuras \ref{fig:conv_train}, \ref{fig:pretrained_train} e \ref{fig:splus_train}. A análise destes dados permite saber o quanto a rede aprende por época de treinamento. Ela é importante para que saibamos por quantas épocas um modelo suporta ser treinado. Sendo assim, uma forma útil de mitigação de \emph{overfiting}\footnote{Excesso de treinamento de uma rede neural articial.}.

% Pela figura \ref{fig:conv_train}, é visto que o \emph{overfiting} começa na oitava época de treinamento. A partir daí, a curva de perda do conjunto de validação começa a divergir da curva de perda do conjunto de treinamento.

% Já na figura \ref{fig:pretrained_train}, é visto que o \emph{overfiting} começa na décima segunda época de treinamento. No entanto, a acurácia final desta rede é menor que da primeira, mesmo com mais épocas de treinamento.

\pagebreak

\begin{figure}[h!]
  \centering
  \includegraphics[width=.79\textwidth]{figures/conv_train2.pdf}
  \caption{Gráfico da acurácia e perda dos subconjuntos de treino e validação em cada época de treinamento do Modelo 1.}
  \label{fig:conv_train}
\end{figure}

\begin{figure}[h!]
  \centering
  \includegraphics[width=.79\textwidth]{figures/pretrained_train2.pdf}
  \caption{Gráfico da acurácia e perda dos subconjuntos de treino e validação em cada época de treinamento do Modelo 2.}
  \label{fig:pretrained_train}
\end{figure}

\begin{figure}[h!]
  \centering
  \includegraphics[width=.79\textwidth]{figures/splus_train.pdf}
  \caption{Gráfico da acurácia e perda dos subconjuntos de treino e validação em cada época de treinamento do Modelo 3.}
  \label{fig:splus_train}
\end{figure}
