\section*{Introdução}

Classificação morfológica é o agrupamento de galáxias conforme sua forma. Como esta classificação é baseada na inspeção visual das imagens, elementos subjetivos são agregados. Em 1926, o astrônomo Edwin Hubble, na tentativa de relacionar a origem das formas das galáxias sistematicamente, criou o método hoje conhecido como \emph{Hubble Sequence} ou \emph{Tunning Fork}. \cite{hubble1926, fortson2012}

\begin{figure}[h!]
    \centering
    \includegraphics[width=.8\textwidth]{figures/tuningfork1.jpg}
    \caption{Diagrama da Sequência de Hubble}
    \label{fig:tuningfork}
\end{figure}

Sabemos que as formas predominantes de grandes galáxias na natureza são elípticas e espirais. A figura \ref{fig:tuningfork} mostra um esboço da atribuição de classes discretas às galáxias de acordo com suas formas. Nela as galáxias são classificadas como elípticas, espirais ou lenticulares. \cite{fortson2012}

O GalaxyZoo\footnote{https://galaxyzoo.org} é um projeto \emph{citizen science} realizado com auxilio de cidadãos, na maioria, sem vínculo acadêmico que contribuem com seus conhecimentos e observações. A segunda liberação de dados do GalaxyZoo possui um catálogo com classificações morfológicas de 300 mil imagens de galáxias do SDSS\footnote{SDSS: Sloan Digital Sky Survey -- https://www.sdss.org} feitas por voluntários e acuradas segundo o método de Hart et al. \cite{hart2016}

Com o avanço dos levantamentos (\emph{surveys}) digitais é substancial a criação de métodos rápidos e automatizados para classificação morfológica de galáxias sem a perda da acurácia da tradicional classificação visual. \cite{yamauchi2005} O uso de \emph{Deep Learning} tem mostrado bons resultados e as Redes Neurais Convolucionais são ferramentas de grande potencial para processamento de imagem de galáxias. \cite{barchi2020, dai2018}

O S-PLUS \cite{oliveira2019} é um levantamento de galáxias do espaço local que observa objetos no campo da Stripe-82.