\section*{Materiais e Métodos}

%%%%%%%%%%%%%%%%%%%%%%%%%%%%%%%%%%%%%%%%%%%
%% VISÃO GERAL
%%%%%%%%%%%%%%%%%%%%%%%%%%%%%%%%%%%%%%%%%%%
\subsection*{Visão Geral}

\begin{figure}[h!]
  \centering
  \includegraphics[width=\textwidth]{figures/dl_diagram.pdf}
  \caption{Fluxograma do desenvolvimento do modelo.}
  \label{fig:fluxogram}
\end{figure}

Pelo fluxograma da figura \ref{fig:fluxogram}, vemos que a primeira etapa no desenvolvimento da Rede Neural Convolucional é a aquisição e o pré-processamento dos dados. A criação da estrutura (sequência de camadas da rede neural) e o ajuste dos hiperparâmetros, também chamdo de \emph{fine tunning}, vem em seguida. É o treimento que une a primeira e a segunda etapa, pois é nele que os dados pré-processados são inseridos na estrutura modelada. Logo em seguida, é feita uma validação da rede treinada. É o desempenho da modelagem anterior que dá o \emph{feedback} das modificações necessárias para a próxima versão da estrutura. O desenvolvimento do modelo termina quando é atingida uma boa acurácia de predição.

%%%%%%%%%%%%%%%%%%%%%%%%%%%%%%%%%%%%%%%%%%%%%
%% AQUISIÇÃO DOS DADOS
%%%%%%%%%%%%%%%%%%%%%%%%%%%%%%%%%%%%%%%%%%%%%
\subsection*{Aquisição dos dados}

Dados de treinamento são elementos fundamentais para o treinamento supervisionado de uma Rede Neural Artificial. Para treinar uma rede que aprenda a classificar galáxias de acordo com suas imagens é necessário fazê-la aprender formas e padrões das galáxias. Para isso, é necessário separar uma grande amostra de imagens de galáxias já classificadas por humanos.

A fonte dos dados são GalaxyZoo, que contém as classificações morfológicas, e SDSS, que possui as imagens das galáxias. A associação dos dados entre estas duas fontes são feitas pelas coordenadas do objeto no espaço.

\begin{figure}[h!]
  \centering
  \includegraphics[width=\textwidth]{figures/galaxy_grid.jpg}
  \caption{Amostra com cinquenta imagens de galáxias (escala reduzida).}
  \label{fig:galaxy_grid}
\end{figure}

O catálogo com as classificações estão em um arquivo CSV. Já as imagens foram obtidas usando a \emph{API}\footnote{API: \emph{Application Programming Interface}; interface de comicação entre sistemas.} do SDSS\footnote{Documentação da API: http://skyserver.sdss.org/dr16/en/help/docs/api.aspx.}. Para acessar a imagem de uma galáxia, o \emph{endpoint}\footnote{O endpoint é o ponto final de um canal de comunicação, ou seja, é a URL acessada para executar alguma função no servidor.} da API do SDSS requere pelo menos dois parâmetros: a ascensão reta\footnote{Distância angular de um ponto específico medido para o leste ao longo do equador celeste.} (RA) e a declinação\footnote{Ângulo que localiza um ponto na esfera celeste no sistema de coordenadas equatoriais.} (DEC).

O algorítmo baixa todas as imagens de galáxias que estão no catálogo do GalaxyZoo usando a API do SDSS. A figura \ref{fig:galaxy_grid} contém uma amostra de cinquenta imagens de galáxias obtidas pelo algorítmo.

Foram baixadas 108.286 imagens no formato JPG e modelo de cores RGB, cada uma com dimensão de 200x200 pixels e escala de 1,35 arsec/pixel do objeto.

%%%%%%%%%%%%%%%%%%%%%%%%%%%%%%%%%%%%%%%%%%%%%%
%% PRÉ-PRECESSAMENTO DOS DADOS
%%%%%%%%%%%%%%%%%%%%%%%%%%%%%%%%%%%%%%%%%%%%%%
\subsection*{Pré-processamento dos dados}

O pré-processamento é a preparação das imagens para serem usadas pelo modelo durante o treinameto. As imagens são separadas em cinco classes: \texttt{Ei}, \texttt{Er}, \texttt{Ec}, \texttt{Ser}, \texttt{Sc2m} e foram agrupadas em três subconjuntos: treinamento, validação e teste.

\begin{table}[h!]
  \centering
  \renewcommand{\arraystretch}{1.6}
  \begin{tabular}{ccccc}
    \toprule
    \thead{Classe} & \thead{Treinamento\\(82,5\%)} & \thead{Validação\\(15\%)} & \thead{Teste\\(2,5\%)} & \thead{Total} \\ 
    \midrule
    Ei      & 36.333    & 6.605     & 1.100     & 44.038 \\
    Er      & 30.331    & 5.514     & 919       & 36.764 \\
    Ser     & 11.558    & 2.101     & 350       & 14.009 \\
    Ec      & 8.374     & 1.522     & 253       & 10.149 \\
    Sc2m    & 2.745     & 498       & 83        & 3.326 \\ 
    \textbf{Total} & \textbf{89.341} & \textbf{16.240} & \textbf{2.705} & \textbf{108.286} \\
    \bottomrule
  \end{tabular}
  \caption{Quantidade de amostras de imagens de galáxias por classe.}
  \label{tab:img_qtd}
\end{table}

Para cada subconjunto é separada uma quantidade de amostras relativa à quantidade total de cada classe: sendo 82,5\% para o treino, 15\% para a validação e 2,5\% para o teste. A tabela \ref{tab:img_qtd} mostra a quantidade de imagens para cada um dos subconjuntos conforme a classe.

Depois de separadas, as imagens precisam ser transformadas em tensores. No sistema RGB, cada cor é representada por três números inteiros no intervalo de 0 a 255. Cada imagem, então, é representada por um tensor de dimensão (200, 200, 3). As primeiras duas dimensões são referentes ao comprimento e largura da imagem e a terceira aos três canais de cores RGB.

Representar uma cor por um terno faz sentido quando se conhece a especificação do sistema de cores RGB, mas isso não é intuituvo para um modelo de rede neural artificial. Por isso, cada número do tensor é reescalado para um número real no intervalo entre zero e um dividindo cada número por 255.

%%%%%%%%%%%%%%%%%%%%%%%%%%%%%%%%%%%%%%%%%%%%%%
%% MODELAGEM DA REDE NEURAL ARTIFICIAL
%%%%%%%%%%%%%%%%%%%%%%%%%%%%%%%%%%%%%%%%%%%%%%
\subsection*{Modelagem e Ajuste dos Hiperparâmetros}

A rede neural artificial foi programada em Python e a biblioteca Keras\footnote{Keras é uma API de alto nível criada para modelagem e treinamento de redes neurais que, neste projeto, roda sobre o TensorFlow. Tem foco em permitir experimentação rápida.} foi usada para criar e treinar o modelo.

A modelagem é a criação de uma estrutura. Isto inclui: a escolha da quantidade de camadas, a definição do tipo de cada camada e a especificação da sequência em que as camadas aparecerão. Já os hiperparâmetros são os atributos de cada camada ou do modelo em geral. Alguns exemplos de hiperparâmetros são: a curva de aprendizagem\footnote{A velocidade com que a rede neural aprende os padrões.} (\emph{learning rate}), a quantidade de unidades\footnote{A quantidade de neurônios artificais.} de cada camada e a taxa (\emph{rate}) das camadas \emph{dropout}. Estes ajustes são feitos empiricamente e reajustados de acordo com o resultado obtido após o treinamento, como visto na figura \ref{fig:fluxogram}.

\pagebreak

\begin{figure}[h!]
  \centering
  \begin{minipage}[t]{.47\textwidth}
    \centering
    \tikzset{
  >=stealth',
  punktchain/.style={
      rectangle,
      rounded corners,
      % fill=black!10,
      draw=black, thick,
      text width=8em,
      minimum height=2em,
      text centered,
      on chain},
  line/.style={draw, thick, <-},
  element/.style={
      tape,
      top color=white,
      bottom color=blue!50!black!60!,
      minimum width=8em,
      draw=blue!40!black!90, very thick,
      text width=8em,
      minimum height=2em,
      text centered,
      on chain},
  every join/.style={->, thick, shorten >=1pt},
}

\begin{tikzpicture}
  [node distance=.5cm,
    start chain=going below,]
  \node[punktchain, join] (input) {\small Input};
  \node[punktchain, join] (conv_1) {\small Conv2D};
  \node[punktchain, join] (maxpool_1) {\small MaxPooling2D};
  \node[punktchain, join] (conv_2) {\small Conv2D};
  \node[punktchain, join] (maxpool_2) {\small MaxPooling2D};
  \node[punktchain, join] (conv_3) {\small Conv2D};
  \node[punktchain, join] (maxpool_3) {\small MaxPooling2D};
  \node[punktchain, join] (flatten) {\small Flatten};
  \node[punktchain, join] (dense_1) {\small Dense};
  \node[punktchain, join] (dropout_1) {\small Dropout};
  % \node[punktchain, join] (dense_2) {\small Dense};
  % \node[punktchain, join] (dropout_2) {\small Dropout};
  \node[punktchain, join] (dense_2) {\small Dense};
\end{tikzpicture}
    \captionof{figure}{\textbf{Modelo 1} -- Camadas convolucionais intercaladas com camadas \emph{pooling}.}
    \label{fig:conv_model}
  \end{minipage}%
  \hfill%
  \begin{minipage}[t]{.47\textwidth}
    \centering
    \tikzset{
  >=stealth',
  punktchain/.style={
      rectangle,
      rounded corners,
      draw=black, thick,
      text width=8em,
      minimum height=2em,
      text centered,
      on chain},
  line/.style={draw, thick, <-},
  every join/.style={->, thick, shorten >=1pt},
}

\begin{tikzpicture}
  [node distance=.5cm,
    start chain=going below,]
  \node[punktchain, join] (input) {\small Input};
  \node[punktchain, join] (vgg16) {\small VGG16};
  \node[punktchain, join] (flatten) {\small Flatten};
  \node[punktchain, join] (dense_1) {\small Dense};
  \node[punktchain, join] (dropout_1) {\small Dropout};
  \node[punktchain, join] (dense_2) {\small Dense};
  \node[punktchain, join] (dropout_2) {\small Dropout};
  \node[punktchain, join] (dense_3) {\small Dense};
  \node[punktchain, join] (dropout_3) {\small Dropout};
  % \node[punktchain, join] (dense_4) {\small Dense};
  % \node[punktchain, join] (dropout_4) {\small Dropout};
  \node[punktchain, join] (dense_5) {\small Dense};
\end{tikzpicture}
    \captionof{figure}{\textbf{Modelo 2} -- Rede pré-treinada ligada à uma rede densa com \emph{dropout}.}
    \label{fig:pretrained_model}
  \end{minipage}
  \end{figure}

\pagebreak

Foram criados dois modelos: um, usando apenas camadas convolucionais e \emph{pooling} ligadas a uma pequena rede densa (figura \ref{fig:conv_model}) e outro, usando uma rede pré-treinada, VGG16, ligada a uma longa rede densa com \emph{dropout} de acordo com o diagrama da figura \ref{fig:pretrained_model}.

Para ambos modelos, a dimensão do tensor da entrada é (200, 200, 3). Cinco unidades na camada de saída com ativador \texttt{Softmax}. A  função de perda utilizada foi a \texttt{Categorical Crossentropy} e o otimizador foi o \texttt{RMSprop}.

\begin{table}[h!]
  \centering
  \begin{tabular}{cc}
    \toprule
    \thead{Hiperparâmetro} & \thead{Valor} \\
    \midrule
    Ativador (Conv2D) & ReLu \\
    Ativador (Dense, 1) & ReLu \\
    Unidades (Conv2D) & 32, 64, 128 e 128 \\
    Unidades (Dense, 1) & 512 \\
    Dimensão do Kernel (Conv2D) & (3, 3) \\
    \emph{Pool Size} (MaxPooling2D) & (2, 2) \\
    Taxa de aprendizagem & $10^{-4}$ \\
    \bottomrule
  \end{tabular}
  \caption{Descrição dos hiperparâmetros do Modelo 1.}
  \label{tab:hip_model1}
\end{table}

\begin{table}[h!]
  \centering
  \begin{tabular}{cc}
    \toprule
    \thead{Hiperparâmetro} & \thead{Valor} \\
    \midrule
    Ativador (Dense, 1-4) & ReLu \\
    Unidades (Dense, 1-4) & 2048, 1024, 512, 256 \\
    Taxa de \emph{dropout} & 0,5 \\
    Taxa de aprendizagem & $10^{-5}$ \\
    \bottomrule
  \end{tabular}
  \caption{Descrição dos hiperparâmetros do Modelo 2.}
  \label{tab:hip_model2}
\end{table}

Como mostrado na figura \ref{fig:fluxogram}, diversos valores de hiperparâmetros foram ajustados empiricamente. As tabelas \ref{tab:hip_model1} e \ref{tab:hip_model2} mostram os valores dos hiperparâmetros com melhor desempenho e que serão discutidos aqui.