\section*{Resumo}
 
Por todo universo, existem galáxias de variadas formas. Estudado a estrutura das galáxias, é possível obter informações importantes desde  sua origem até sua evolução. A classificação morfológica é a catalogação de galáxias de acordo com sua aparência visual. Contudo, a análise visual implica em viés causado pela subjetividade da observação humana. Por isso, a classificação sistemática de galáxias vem ganhando importância desde quando o astrônomo Edwin Hubble criou seu método classificação. Neste trabalho, nós combinamos classificações visuais acuradas do projeto Galaxy Zoo com métodos de \emph{Deep Learning}. Foram criados dois modelos de redes neurais convolucionais usando diferentes técnicas. O objetivo é encontrar técnincas eficientes que consigam simular a classificação visual humana, mas de forma metódica.